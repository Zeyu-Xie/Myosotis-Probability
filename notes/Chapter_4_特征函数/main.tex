\documentclass[12pt,a4paper]{amsart}
\usepackage[UTF8]{ctex}
\usepackage{preamble}
\usepackage{tikz}

% 定义 example 环境
\newcounter{example}[section] % 定义 example 计数器
\newenvironment{example}{%
    \refstepcounter{example}% 增加计数器
    \par\medskip\noindent
    \textbf{Example \theexample:}\ %
}{%
    \par\medskip
}

% 定义 solution 环境
\newcounter{solution}[section] % 定义 solution 计数器
\newenvironment{solution}{%
    \refstepcounter{solution}% 增加计数器
    \par\medskip\noindent
    \textbf{Solution \thesolution:}\ %
}{%
    \par\medskip
}

\title{Chapter 4 特征函数}

\begin{document}

\maketitle\cite{杨振明2007}

\section{母函数}

\begin{definition}[母函数]
    对任何实数列 $\{p_n\}$,如果幂级数
    \begin{equation}
        G(s) = \sum_{n=0}^{\infty} p_n s^n
    \end{equation}
    的收敛半径 $s_0 > 0$,则称 $G(s)$ 为 $\{p_n\}$ 的母函数。 \\
    特别地,当 $\{p_n\}$ 为某非负整值随机变量 $\xi$ 的概率分布时,$G(s)$ 至少在区间 $[-1, 1]$ 上绝对收敛且一致收敛,此时有
    \begin{equation}
        G(s) = E(s^\xi)
    \end{equation}
    称此 $G(s)$ 为随机变量 $\xi$ 或其概率分布 $\{p_n\}$ 的母函数。
\end{definition}

\begin{example}
    求 Possion 分布和几何分布的母函数。
\end{example}

\begin{solution}
    \begin{enumerate}
        \item Possion 分布的母函数
        \begin{equation}
            G(s) = \sum_{n=0}^{\infty} \frac{\lambda^n}{n!} e^{-\lambda} s^n = e^{\lambda(s-1)}
        \end{equation}
        收敛域为 $(-\infty, +\infty)$。
        \item 几何分布的母函数
        \begin{equation}
            G(s) = \sum_{n=0}^{\infty} (1-p)^{n-1} p s^n = \frac{ps}{1-(1-p)s}
        \end{equation}
        收敛域为 $(-\frac{1}{1-p}, \frac{1}{1-p})$。
    \end{enumerate}
\end{solution}

\begin{proposition}[分布由母函数唯一确定]
    显然母函数由分布唯一确定,反过来说,由于 $G(s)$ 至少可以在区间 $(-1, 1)$ 内逐项求导,再令 $s=0$ 得
    \begin{equation}
        p_n = \frac{1}{n!} G^{(n)}(0)
    \end{equation}
    所以母函数 $G(s)$ 也可以由分布 $\{p_n\}$ 唯一确定。
\end{proposition}

\begin{proposition}[母函数与数学期望、方差的关系]
    设非负整值随机变量 $\xi$ 的母函数为 $G(s)$,如果 $E(\xi)$ 和 $E(\xi^2)$ 有限,那么
    \begin{equation}
        \begin{aligned}
            G'(1) &= E(\xi) \\
            G''(1) &= E(\xi^2) - E(\xi)
        \end{aligned}
    \end{equation}
\end{proposition}

\begin{proposition}[独立和的母函数]
    设 $\xi$ 和 $\eta$ 是两个独立的非负整值随机变量,分别有概率分布 $\{a_n\}$ 和 $\{b_n\}$,母函数为 $A(s)$ 和 $B(s)$,则 $\xi + \eta$ 的母函数为
    \begin{equation}
        C(s) = A(s)B(s)
    \end{equation}
\end{proposition}

\begin{proposition}[随机多个非负整值随机变量之和的母函数]
    设 $\{\xi_k\}$ 为 相互独立的非负整值随机变量序列,有共同的母函数 $G(s)$。若 $\eta$ 为另一非负整值随机变量,其母函数为 $F(s)$,那么当 $\eta$ 与每个 $\xi_k$ 均独立时,$\xi = \sum_{k=1}^\eta \xi_k$ 的母函数为
    \begin{equation}
        H(s) = F[G(s)]
    \end{equation}
\end{proposition}

\section{特征函数}

\begin{definition}[特征函数]
    设 $F(x)$ 为 $\R = (-\infty, +\infty)$ 上的一个分布函数,称
    \begin{equation}
        f(t) = \int_{-\infty}^{+\infty} e^{itx} dF(x)
    \end{equation}
    为 $F(x)$ 的特征函数。
\end{definition}

\begin{definition}[随机变量的特征函数]
    设 $F(x)$ 为随机变量 $\xi$ 的分布函数,则此 $f(t)$ 也称为 $\xi$ 的特征函数,此时有
    \begin{equation}\label{eq:特征函数的期望}
        f(t) = E(e^{it\xi})
    \end{equation}
\end{definition}

注:对 $\forall ~ t,x\in \R$,总有 $|e^{itx}| = 1$,故 \ref{eq:特征函数的期望} 式右端积分的模不超过 $1$。因此对任意的概率分布,其特征函数唯一确定地存在。\footnote{这就比只对非负整值随机变量有定义的母函数好很多}

\begin{definition}[离散型随机变量的特征函数]
    设 $\xi$ 为离散型随机变量,其概率分布为 $\{p_n\}$,则其特征函数为
    \begin{equation}
        f(t) = \sum_{k=0}^{\infty} p_k e^{itx_k},\quad p_k = P\{\xi = x_k\}
    \end{equation}
\end{definition}

\begin{definition}[连续型随机变量的特征函数]
    设 $\xi$ 为连续型随机变量,其概率密度函数为 $p(x)$,则其特征函数为
    \begin{equation}
        f(t) = \int_{-\infty}^{+\infty} p(x) e^{itx} dx
    \end{equation}
\end{definition}

\section{多元正态分布}

\appendix

\bibliographystyle{unsrt}
{\footnotesize\bibliography{./library}}


\end{document}
