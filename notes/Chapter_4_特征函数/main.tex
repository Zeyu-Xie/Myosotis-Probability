\documentclass[12pt,a4paper]{amsart}
\usepackage[UTF8]{ctex}
\usepackage{preamble}
\usepackage{tikz}

% 定义 example 环境
\newcounter{example}[section] % 定义 example 计数器
\newenvironment{example}{%
    \refstepcounter{example}% 增加计数器
    \par\medskip\noindent
    \textbf{Example \theexample:}\ %
}{%
    \par\medskip
}

% 定义 solution 环境
\newcounter{solution}[section] % 定义 solution 计数器
\newenvironment{solution}{%
    \refstepcounter{solution}% 增加计数器
    \par\medskip\noindent
    \textbf{Solution \thesolution:}\ %
}{%
    \par\medskip
}

\title{Chapter 4 特征函数}

\begin{document}

\maketitle\cite{杨振明2007}

\section{母函数}

\begin{definition}[母函数]
    对任何实数列 $\{p_n\}$,如果幂级数
    \begin{equation}
        G(s) = \sum_{n=0}^{\infty} p_n s^n
    \end{equation}
    的收敛半径 $s_0 > 0$,则称 $G(s)$ 为 $\{p_n\}$ 的母函数。 \\
    特别地,当 $\{p_n\}$ 为某非负整值随机变量 $\xi$ 的概率分布时,$G(s)$ 至少在区间 $[-1, 1]$ 上绝对收敛且一致收敛,此时有
    \begin{equation}
        G(s) = E(s^\xi)
    \end{equation}
    称此 $G(s)$ 为随机变量 $\xi$ 或其概率分布 $\{p_n\}$ 的母函数。
\end{definition}

\section{特征函数}

\section{多元正态分布}

\appendix

\bibliographystyle{unsrt}
{\footnotesize\bibliography{./library}}


\end{document}
