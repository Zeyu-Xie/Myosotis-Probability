\documentclass[12pt,a4paper]{amsart}
\usepackage[UTF8]{ctex}
\usepackage{preamble}
\usepackage{tikz}

% 定义 example 环境
\newcounter{example}[section] % 定义 example 计数器
\newenvironment{example}{%
    \refstepcounter{example}% 增加计数器
    \par\medskip\noindent
    \textbf{Example \theexample:}\ %
}{%
    \par\medskip
}

% 定义 solution 环境
\newcounter{solution}[section] % 定义 solution 计数器
\newenvironment{solution}{%
    \refstepcounter{solution}% 增加计数器
    \par\medskip\noindent
    \textbf{Solution \thesolution:}\ %
}{%
    \par\medskip
}

\title{Chapter 4 特征函数}

\begin{document}

\maketitle\cite{杨振明2007}

\section{母函数}

\begin{definition}[母函数]
    对任何实数列 $\{p_n\}$,如果幂级数
    \begin{equation}
        G(s) = \sum_{n=0}^{\infty} p_n s^n
    \end{equation}
    的收敛半径 $s_0 > 0$,则称 $G(s)$ 为 $\{p_n\}$ 的母函数。 \\
    特别地,当 $\{p_n\}$ 为某非负整值随机变量 $\xi$ 的概率分布时,$G(s)$ 至少在区间 $[-1, 1]$ 上绝对收敛且一致收敛,此时有
    \begin{equation}
        G(s) = E(s^\xi)
    \end{equation}
    称此 $G(s)$ 为随机变量 $\xi$ 或其概率分布 $\{p_n\}$ 的母函数。
\end{definition}

\begin{example}
    求 Possion 分布和几何分布的母函数。
\end{example}

\begin{solution}
    \begin{enumerate}
        \item Possion 分布的母函数
        \begin{equation}
            G(s) = \sum_{n=0}^{\infty} \frac{\lambda^n}{n!} e^{-\lambda} s^n = e^{\lambda(s-1)}
        \end{equation}
        收敛域为 $(-\infty, +\infty)$。
        \item 几何分布的母函数
        \begin{equation}
            G(s) = \sum_{n=0}^{\infty} (1-p)^{n-1} p s^n = \frac{ps}{1-(1-p)s}
        \end{equation}
        收敛域为 $(-\frac{1}{1-p}, \frac{1}{1-p})$。
    \end{enumerate}
\end{solution}

\begin{proposition}[分布由母函数唯一确定]
    显然母函数由分布唯一确定,反过来说,由于 $G(s)$ 至少可以在区间 $(-1, 1)$ 内逐项求导,再令 $s=0$ 得
    \begin{equation}
        p_n = \frac{1}{n!} G^{(n)}(0)
    \end{equation}
    所以母函数 $G(s)$ 也可以由分布 $\{p_n\}$ 唯一确定。
\end{proposition}

\begin{proposition}[母函数与数学期望、方差的关系]
    设非负整值随机变量 $\xi$ 的母函数为 $G(s)$,如果 $E(\xi)$ 和 $E(\xi^2)$ 有限,那么
    \begin{equation}
        \begin{aligned}
            G'(1) &= E(\xi) \\
            G''(1) &= E(\xi^2) - E(\xi)
        \end{aligned}
    \end{equation}
\end{proposition}

\begin{proposition}[独立和的母函数]
    设 $\xi$ 和 $\eta$ 是两个独立的非负整值随机变量,分别有概率分布 $\{a_n\}$ 和 $\{b_n\}$,母函数为 $A(s)$ 和 $B(s)$,则 $\xi + \eta$ 的母函数为
    \begin{equation}
        C(s) = A(s)B(s)
    \end{equation}
\end{proposition}

\begin{proposition}[随机多个非负整值随机变量之和的母函数]
    设 $\{\xi_k\}$ 为 相互独立的非负整值随机变量序列,有共同的母函数 $G(s)$。若 $\eta$ 为另一非负整值随机变量,其母函数为 $F(s)$,那么当 $\eta$ 与每个 $\xi_k$ 均独立时,$\xi = \sum_{k=1}^\eta \xi_k$ 的母函数为
    \begin{equation}
        H(s) = F[G(s)]
    \end{equation}
\end{proposition}

\section{特征函数}

\begin{definition}[特征函数]
    设 $F(x)$ 为 $\R = (-\infty, +\infty)$ 上的一个分布函数,称
    \begin{equation}
        f(t) = \int_{-\infty}^{+\infty} e^{itx} dF(x)
    \end{equation}
    为 $F(x)$ 的特征函数。
\end{definition}

\begin{definition}[随机变量的特征函数]
    设 $F(x)$ 为随机变量 $\xi$ 的分布函数,则此 $f(t)$ 也称为 $\xi$ 的特征函数,此时有
    \begin{equation}\label{eq:特征函数的期望}
        f(t) = E(e^{it\xi})
    \end{equation}
\end{definition}

注:对 $\forall ~ t,x\in \R$,总有 $|e^{itx}| = 1$,故 \ref{eq:特征函数的期望} 式右端积分的模不超过 $1$。因此对任意的概率分布,其特征函数唯一确定地存在。\footnote{这就比只对非负整值随机变量有定义的母函数好很多}

\begin{definition}[离散型随机变量的特征函数]
    设 $\xi$ 为离散型随机变量,其概率分布为 $\{p_n\}$,则其特征函数为
    \begin{equation}
        f(t) = \sum_{k=0}^{\infty} p_k e^{itx_k},\quad p_k = P\{\xi = x_k\}
    \end{equation}
\end{definition}

\begin{definition}[连续型随机变量的特征函数]
    设 $\xi$ 为连续型随机变量,其概率密度函数为 $p(x)$,则其特征函数为
    \begin{equation}
        f(t) = \int_{-\infty}^{+\infty} p(x) e^{itx} dx
    \end{equation}
\end{definition}

\begin{proposition}[各种分布的特征函数]
    离散型随机变量的特征函数
    \begin{table}[H]
        \centering
        \caption{离散型随机变量的特征函数}
        \begin{tabular}{cc}
            \toprule
            分布类型 & 特征函数                            \\
            \midrule
            Bernoulli 分布     & $f(t) = q+pe^{it}$                             \\
            二项分布           & $f(t) = (q+pe^{it})^n$                         \\
            几何分布           & $f(t) = \frac{pe^{it}}{1-(1-p)e^{it}}$              \\
            Pascal 分布       & $f(t) = \left[\frac{pe^{it}}{1-(1-p)e^{it}}\right]^n$ \\
            Possion 分布       & $f(t) = e^{\lambda(e^{it}-1)}$                \\
            \bottomrule
        \end{tabular}
        \label{chart:1}
    \end{table}
    连续型随机变量的特征函数
    \begin{table}[H]
        \centering
        \caption{连续型随机变量的特征函数}
        \begin{tabular}{cc}
            \toprule
            分布类型 & 特征函数                            \\
            \midrule
            正态分布           & $f(t) = e^{i\mu t-\frac{1}{2}\sigma^2t^2}$ \\
            Gamma 分布         & $f(t) = (1-\frac{it}{\lambda})^{-\alpha}$ \\
            指数分布           & $f(t) = \frac{\lambda}{\lambda - it}$ \\
            均匀分布           & $f(t) = \frac{e^{itb}-e^{ita}}{it(b-a)}$ \\
            \bottomrule
        \end{tabular}
        \label{chart:2}
    \end{table}
\end{proposition}

\begin{proposition}[特征函数基本性质]
    \begin{enumerate}
        \item $|f(t)|\leq f(0) = 1$
        \item 共轭对称性:$f(-t) = \overline{f(t)}$
        \item $f(t)$ 在 $t\in \R$ 上一致连续
        \item 半正定性:任意 $n\geq 1$,任意 $n$ 个实数 $t_1, t_2, \cdots, t_n$,任意 $n$ 个复数 $a_1, a_2, \cdots, a_n$,有
        \begin{equation}
            \sum_{i=1}^{n} \sum_{j=1}^{n} a_i \overline{a_j} f(t_i - t_j) \geq 0
        \end{equation}
        \item $f_{a+b\xi}(t) = e^{iat}f_{\xi}(bt)$
    \end{enumerate}
\end{proposition}

\begin{proposition}
    如果随机变量 $\xi$ 的各阶原点矩有限,那么对一切满足
    \begin{equation}
        \lim\limits_{n\to\infty} \frac{|t|^nE|\xi|^n}{n!} = 0
    \end{equation}
    的 $t$,$\xi$ 的特征函数 $f(t)$ 有展开式
    \begin{equation}
        f(t) = \sum_{k=0}^{\infty} \frac{(it)^k}{k!} E(\xi^k)
    \end{equation}
    该定理可用于计算符合某些条件的概率分布的特征函数。
\end{proposition}

\begin{proposition}
    设随机变量 $\xi$ 的 $k$ 阶原点矩有限,则其特征函数 $k$ 阶可微,且有
    \begin{equation}
        f^{(k)}(t) = E\left[(i\xi)^k e^{it\xi}\right]
    \end{equation}
\end{proposition}

\begin{proposition}[独立随机变量之和]
    设 $\xi$ 和 $\eta$ 为两个独立的随机变量,其特征函数分别为 $f_{\xi}(t)$ 和 $f_{\eta}(t)$,则 $\xi + \eta$ 的特征函数为
    \begin{equation}
        f_{\xi+\eta}(t) = f_{\xi}(t)f_{\eta}(t)
    \end{equation}
    即独立随机变量之和的特征函数等于各自特征函数的乘积。
\end{proposition}

\section{反演公式与唯一性定理}

\begin{lemma}
    \begin{equation}
        \lim\limits_{c\to+\infty} \int_0^c \frac{sin ax}{x} dx = \frac{\pi}{2} sgn\{a\}
    \end{equation}
\end{lemma}

\begin{theorem}[反演公式]
    设 $f(t)$ 为随机变量 $\xi$ 的特征函数,$f(t)$ 在 $t$ 的某个邻域内连续,且 $f(t)$ 在 $t=0$ 处连续,且 $f(0) = 1$,则 $\xi$ 的分布函数 $F(x)$ 可以由 $f(t)$ 唯一确定,且有
    \begin{equation}
        F(x) = \frac{1}{2\pi} \int_{-\infty}^{+\infty} e^{-itx} f(t) dt
    \end{equation}
\end{theorem}

\begin{theorem}[唯一性定理]\label{thm:唯一性定理}
    分布函数由其特征函数唯一确定。
\end{theorem}

\begin{theorem}
    如果特征函数的模可积,即
    \begin{equation}
        \int_{-\infty}^{+\infty} |f(t)| dt < +\infty
    \end{equation}
    那么对应的分布函数 $F(x)$ 为连续型,且其密度函数为
    \begin{equation}
        p(x) = \frac{1}{2\pi} \int_{-\infty}^{+\infty} e^{-itx} f(t) dt
    \end{equation}
\end{theorem}

\begin{definition}[再生性]
    设 $F(x;c)$ 为分布函数,其中 $c$ 为此概率分布的参数,如果有
    \begin{equation}
        F(x; c_1) * F(x; c_2) = F(x; c_1+c_2)
    \end{equation}
    则称此分布关于参数 $c$ 具有再生性。 \\
    由 \ref{thm:唯一性定理} 可知,特征函数的乘积等于特征函数的和,所以再生性也可以由特征函数的乘积等于特征函数的和来刻画
    \begin{equation}
        f(t; c_1) f(t; c_2) = f(t; c_1+c_2)
    \end{equation}
\end{definition}

\begin{definition}[多元特征函数]
    定义 $n$ 元分布函数 $F(x_1, x_2, \cdots, x_n)$ 所对应的 $n$ 元特征函数为
    \begin{equation}
        f(t_1, t_2, \cdots, t_n) = \int_{-\infty}^{+\infty} \cdots \int_{-\infty}^{+\infty} e^{i(t_1x_1+t_2x_2+\cdots+t_nx_n)} dF(x_1, x_2, \cdots, x_n)
    \end{equation}
    如果 $F(x_1, x_2, \cdots, x_n)$ 是随机向量 $(\xi_1, \xi_2, \cdots, \xi_n)$ 的分布函数,则有
    \begin{equation}
        f(t_1, t_2, \cdots, t_n) = E\left[e^{i(t_1\xi_1+t_2\xi_2+\cdots+t_n\xi_n)}\right]
    \end{equation}
    此时称 $f$ 为随机向量 $(\xi_1, \xi_2, \cdots, \xi_n)$ 的联合特征函数,它在 $\R^n$ 上是一致连续的。
\end{definition}

\begin{proposition}[多元特征函数的性质]
    \begin{enumerate}
        \item $|f(t_1, t_2, \cdots, t_n)| \leq f(0, 0, \cdots, 0) = 1$
        \item $f(-t_1, -t_2, \cdots, -t_n) = \overline{f(t_1, t_2, \cdots, t_n)}$
        \item $f(t_1, t_2, \cdots, t_n)$ 在 $\R^n$ 上一致连续
        \item 若混合矩 $E(\xi_1^{k_1}\xi_2^{k_2}\cdots\xi_n^{k_n})$ 有限,则可用 $(\xi_1, \xi_2, \cdots, \xi_n)$ 的联合特征函数在原点的矩阵求导,即有公式
        \begin{equation}
            f^{(k_1, k_2, \cdots, k_n)} = i^{-\sum_{j=1}^{n} k_j} \frac{\partial^{k_1+k_2+\cdots+k_n} f(t_1, t_2, \cdots, t_n)}{\partial t_1^{k_1} \partial t_2^{k_2} \cdots \partial t_n^{k_n}} \Bigg|_{t_1=t_2=\cdots=t_n=0}
        \end{equation}
        \item (反演公式)如果随机向量 $(\xi_1, \xi_2, \cdots, \xi_n)$ 取值于 $n$ 维区间 $\Delta = \{(x_1, x_2, \cdots, x_n) | a_i \leq x_i \leq b_i, i=1, 2, \cdots, n\}$ 的边界面上的概率为 0,则有
        \begin{equation}
            \begin{aligned}
                &P\{(\xi_1, \xi_2, \cdots, \xi_n)\in\Delta\} \\
                =&P\left\{\cap_{j=1}^{n} [a_j \leq \xi_j \leq b_j] \right\} \\
                =&\lim\limits_{\substack{c_j\to+\infty \\ j = 1,2,\cdots,n}} \frac{1}{(2\pi)^n} \int_{-c_1}^{c_1} \int_{-c_2}^{c_2} \cdots \int_{-c_n}^{c_n} \prod_{j=1}^{n} \frac{e^{-it_ja_j}-e^{-it_jb_j}}{it_j} f(t_1, t_2, \cdots, t_n) dt_1 dt_2 \cdots dt_n
            \end{aligned}
        \end{equation}
        \item 多元分布函数和特征函数也是一一对应的。
    \end{enumerate}
\end{proposition}

\section{多元正态分布}

\appendix

\bibliographystyle{unsrt}
{\footnotesize\bibliography{./library}}


\end{document}
