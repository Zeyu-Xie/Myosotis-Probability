\documentclass[12pt,a4paper]{amsart}
\usepackage[UTF8]{ctex}
\usepackage{preamble}

% 定义 example 环境
\newcounter{example}[section] % 定义 example 计数器
\newenvironment{example}{%
    \refstepcounter{example}% 增加计数器
    \par\medskip\noindent
    \textbf{Example \theexample:}\ %
}{%
    \par\medskip
}

% 定义 solution 环境
\newcounter{solution}[section] % 定义 solution 计数器
\newenvironment{solution}{%
    \refstepcounter{solution}% 增加计数器
    \par\medskip\noindent
    \textbf{Solution \thesolution:}\ %
}{%
    \par\medskip
}

\title{Chapter 3 数学特征与特征函数}

\begin{document}

\maketitle\cite{杨振明2007}

\section{数学期望}

\subsection{定义}

\begin{definition}[离散型随机变量的数学期望]
    设离散型随机变量 $\xi$ 的概率分布为 $p_i = P\{\xi = x_i\}$,$i = 1, 2, \ldots$,若 $\sum_{i} |x_i|p_i < +\infty$,则称
    \begin{equation}
        E(\xi) = \sum_{i} x_i p_i
    \end{equation}
    为随机变量 $\xi$ 的数学期望。
\end{definition}

\begin{definition}[连续型随机变量的数学期望]
    设连续型随机变量 $\xi$ 的概率密度为 $p(x)$,若 $\int_{-\infty}^{+\infty} |x|p(x) dx < +\infty$,则称
    \begin{equation}
        E(\xi) = \int_{-\infty}^{+\infty} x p(x) dx
    \end{equation}
    为随机变量 $\xi$ 的数学期望。
\end{definition}

\begin{definition}[数学期望的统一写法]
    设 $\xi$ 为随机变量,则定义
    \begin{equation}
        E(\xi) = \int_{-\infty}^{+\infty} x dF(x)
    \end{equation}
    为随机变量 $\xi$ 的数学期望。
\end{definition}

\subsection{基本性质}

\begin{proposition}[数学期望的性质]
    设 $\xi, \eta$ 为随机变量,且都有有限的数学期望,则有
    \begin{enumerate}
        \item $E(c) = c$
        \item $E(c\xi) = cE(\xi)$
        \item $E(\xi + \eta) = E(\xi) + E(\eta)$
        \item 若 $\xi \geq 0$,则 $E(\xi) \geq 0$
        \item 若 $\xi \geq \eta$,则 $E(\xi) \geq E(\eta)$
    \end{enumerate}
\end{proposition}

\begin{proposition}[Borel 函数下的数学期望]
    设 $\xi$ 为随机变量,$f(x)$ 为 Borel 函数,则有
    \begin{equation}
        E[f(\xi)] = \int_{-\infty}^{+\infty} f(x) dF(x)
    \end{equation}
    此性质在多元随机变量的情况下也成立:设随机变量 $(\xi_1, \xi_2, \cdots, \xi_n)$ 有联合分布函数 $F(x_1, x_2, \cdots, x_n)$,$f$ 为 $n$ 元 Borel 函数,则有
    \begin{equation}
        E[f(\xi_1, \xi_2, \cdots, \xi_n)] = \int_{-\infty}^{+\infty} \cdots \int_{-\infty}^{+\infty} f(x_1, x_2, \cdots, x_n) dF(x_1, x_2, \cdots, x_n)
    \end{equation}
\end{proposition}

注:以上性质说明,可直接用 $(\xi_1, \xi_2, \cdots, \xi_n)$ 的联合分布计算 $\eta = f(\xi_1, \xi_2, \cdots, \xi_n)$ 的数学期望,而不必先求出 $\eta$ 的分布。

\subsection{独立随机变量的性质}

\begin{proposition}[独立随机变量数学期望的性质]
    设 $\xi, \eta$ 为独立随机变量,且 $\xi$ 与 $\eta$ 均可积,则乘积 $\xi\eta$ 也可积,且有
    \begin{equation}
        E(\xi\eta) = E(\xi)E(\eta)
    \end{equation}
\end{proposition}

\begin{proposition}[独立随机变量数学期望的等价条件]
    概率空间 $(\Omega, \mathcal{F}, P)$ 中的随机变量 $\xi, \eta$ 独立的充要条件是:对任意使得 $f(\xi)$ 和 $g(\eta)$ 可积的 Borel 函数 $f, g$,有
    \begin{equation}
        E[f(\xi)g(\eta)] = E[f(\xi)]E[g(\eta)]
    \end{equation}
\end{proposition}

\subsection{极限性质}

我们考虑

\begin{equation}\label{eq:极限性质}
    \lim\limits_{n\to\infty} E(\xi_n) = E(\lim\limits_{n\to\infty} \xi_n)
\end{equation}

成立的条件。

\begin{proposition}[单调收敛定理]
    设随机变量序列 $\{\xi_n\}$ 满足条件
    \begin{equation}
        0 \leq \xi_1(\omega) \leq \xi_2(\omega) \leq \cdots \leq \xi_n(\omega) ~ \uparrow ~ \xi(\omega), \quad \forall \omega \in \Omega
    \end{equation}
    则 \ref{eq:极限性质} 成立。
\end{proposition}

\begin{proposition}[Fatou 引理]
    设 $\{\xi_n\}$ 是一随机变量序列
    \begin{enumerate}
        \item 若存在可积随机变量 $\sigma$,使得 $\xi_n \geq \sigma$, 则有
        \begin{equation}
            E(\lim\limits_{n\to\infty} \inf \xi_n) \leq \lim\limits_{n\to\infty} \inf E(\xi_n)
        \end{equation}
        \item 若存在可积随机变量 $\tau$,使得 $\xi_n \leq \tau$, 则有
        \begin{equation}
            E(\lim\limits_{n\to\infty} \sup \xi_n) \geq \lim\limits_{n\to\infty} \sup E(\xi_n)
        \end{equation}
    \end{enumerate}
\end{proposition}

\begin{proposition}[Lebesgue 控制收敛定理]
    设 $\{\xi_n\}$ 是一随机变量序列,若存在可积随机变量 $\eta$ 使得 $|\xi_n| \leq \eta$,且 $\lim\limits_{n\to\infty} \xi_n = \xi$,则 \ref{eq:极限性质} 成立,即
    \begin{equation}
        \lim\limits_{n\to\infty} E(\xi_n) = E(\xi)
    \end{equation}
\end{proposition}

\subsection{常见分布的期望}

\begin{proposition}[Bernoulli 分布的数学期望]
    设随机变量 $\xi$ 服从参数为 $p$ 的 Bernoulli 分布,则有
    \begin{equation}
        E(\xi) = p
    \end{equation}
\end{proposition}

\begin{proposition}[二项分布的数学期望]
    设随机变量 $\xi$ 服从参数为 $(n, p)$ 的二项分布,则有
    \begin{equation}
        E(\xi) = np
    \end{equation}
\end{proposition}

\begin{proposition}[Possion 分布的数学期望]
    设随机变量 $\xi$ 服从参数为 $\lambda$ 的 Possion 分布,则有
    \begin{equation}
        E(\xi) = \lambda
    \end{equation}
    证明:直接计算
\end{proposition}

\begin{proposition}[几何分布的数学期望]
    设随机变量 $\xi$ 服从参数为 $p$ 的几何分布,则有
    \begin{equation}
        E(\xi) = \frac{1}{p}
    \end{equation}
\end{proposition}

\begin{proposition}[均匀分布的数学期望]
    设随机变量 $\xi$ 服从参数为 $(a, b)$ 的均匀分布,则有
    \begin{equation}
        E(\xi) = \frac{a + b}{2}
    \end{equation}
\end{proposition}

\begin{proposition}[正态分布的数学期望]
    设随机变量 $\xi$ 服从参数为 $(\mu, \sigma^2)$ 的正态分布,则有
    \begin{equation}
        E(\xi) = \mu
    \end{equation}
\end{proposition}

\begin{proposition}[\( \chi^2 \) 分布的数学期望]
    设随机变量 $\xi$ 服从参数为 $n$ 的 \( \chi^2 \) 分布,则有
    \begin{equation}
        E(\xi) = n
    \end{equation}
\end{proposition}

\begin{proposition}[Cauchy 分布的数学期望]
    设随机变量 $\xi$ 有密度函数
    \begin{equation}
        p(x) = \frac{1}{\pi(1 + x^2)}
    \end{equation}
    注意到
    \begin{equation}
        \int_{-\infty}^{+\infty} |x| p(x) dx = \int_{-\infty}^{+\infty} \frac{|x|}{\pi(1 + x^2)} dx = +\infty
    \end{equation}
    故 Cauchy 分布的数学期望不存在。
\end{proposition}

注:常见的分布中,数学期望不存在的仅有 Cauchy 分布。

\section{方差}

\appendix


\bibliographystyle{unsrt}
{\footnotesize\bibliography{./library}}


\end{document}
