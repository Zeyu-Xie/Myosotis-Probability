\documentclass[12pt,a4paper]{amsart}
\usepackage[UTF8]{ctex}
\usepackage{preamble}

% 定义 example 环境
\newcounter{example}[section] % 定义 example 计数器
\newenvironment{example}{%
    \refstepcounter{example}% 增加计数器
    \par\medskip\noindent
    \textbf{Example \theexample:}\ %
}{%
    \par\medskip
}

% 定义 solution 环境
\newcounter{solution}[section] % 定义 solution 计数器
\newenvironment{solution}{%
    \refstepcounter{solution}% 增加计数器
    \par\medskip\noindent
    \textbf{Solution \thesolution:}\ %
}{%
    \par\medskip
}

\title{Chapter 1 事件与概率}

\begin{document}

\maketitle

\section{概率空间}

\subsection{代数、$\sigma$ 代数、单调类、$\pi$ 类、$\lambda$ 类}

设 $\phi$ 为空间 $\Omega$ 的子集组成的非空类

\begin{definition}[代数和 $\sigma$ 代数]
    对有限交、取余封闭,则称 $\phi$ 为 $\Omega$ 上的代数 \\
    若 $\phi$ 对无限交、取余封闭,则称 $\phi$ 为 $\Omega$ 上的 $\sigma$ 代数
\end{definition}

\begin{definition}[单调类]
    若 $\phi$ 对单调极限封闭,则称 $\phi$ 为 $\Omega$ 上的单调类
\end{definition}

\begin{definition}[$\pi$ 类]
    若 $\phi$ 对有限交封闭,则称 $\phi$ 为 $\Omega$ 上的 $\pi$ 类
\end{definition}

\begin{definition}[$\lambda$ 类]
    若 $\phi$ 对真差运算、上极限封闭,则称 $\phi$ 为 $\Omega$ 上的 $\lambda$ 类
\end{definition}

代数、$\sigma$ 代数、单调类、$\pi$ 类、$\lambda$ 类之间有如下性质

\begin{proposition}
    \begin{enumerate}
        \item $\sigma$ 代数 $\Rightarrow$ 代数 $\Rightarrow$ 单调类
        \item 代数 and 单调类 $\Rightarrow$ $\sigma$ 代数
        \item $\pi$ 类 and $\lambda$ 类 $\Rightarrow$ $\sigma$ 代数
    \end{enumerate}
\end{proposition}

以及如下的单调类定理

\begin{proposition}[单调类定理]\footnote{$m(\phi)$、$\sigma(\phi)$、$\lambda(\phi)$ 分别表示 $\phi$ 生成的最小单调类、最小 $\sigma$ 代数、最小 $\lambda$ 类}
    \begin{enumerate}
        \item 若 $\phi$ 为一代数,则 $m(\phi) = \sigma(\phi)$
        \item 若 $\phi$ 为一 $\pi$ 类,则 $\lambda(\phi) = \sigma(\phi)$
    \end{enumerate}
\end{proposition}

以及如下的另一个定理

\begin{proposition}
    设 $\phi$ 和 $\mathcal{F}$ 为 $\omega$ 中的两个集类,$\phi\subseteq \mathcal{F}$
    \begin{enumerate}
        \item 若 $\phi$ 为代数而 $\mathcal{F}$ 为单调类 $\Rightarrow$ $\sigma(\phi)\subseteq\mathcal{F}$
        \item 若 $\phi$ 为 $\pi$ 类而 $\mathcal{F}$ 为 $\lambda$ 类 $\Rightarrow$ $\sigma(\phi)\subseteq\mathcal{F}$
    \end{enumerate}
\end{proposition}

\subsection{概率空间}

\begin{definition}[概率空间]
    设 $\Omega$ 为样本空间,$\mathcal{F}$ 为 $\Omega$ 上的 $\sigma$ 代数,$P$ 为定义在 $\mathcal{F}$ 上的函数,若满足
    \begin{enumerate}
        \item $P(A)\geq 0,\forall A\in\mathcal{F}$
        \item $P(\Omega) = 1$
        \item 若 $A_1,A_2,\ldots\in\mathcal{F}$ 两两互斥,则 $P(\bigcup_{i=1}^\infty A_i) = \sum_{i=1}^\infty P(A_i)$
    \end{enumerate}
    则称 $(\Omega,\mathcal{F},P)$ 为概率空间
\end{definition}

\subsection{条件概率}

\begin{definition}[条件概率]
    设 $(\Omega,\mathcal{F},P)$ 为概率空间,$B\in\mathcal{F}$ 且 $P(B)>0$,则对任意 $A\in\mathcal{F}$,定义
    \[ P(A|B) = \frac{P(AB)}{P(B)} \]
    为在事件 $B$ 发生的条件下事件 $A$ 发生的条件概率
\end{definition}

条件概率有以下基本性质

\begin{proposition}
    设 $(\Omega,\mathcal{F},P)$ 为概率空间,$B\in\mathcal{F}$ 且 $P(B)>0$,则
    \begin{enumerate}
        \item 乘法定理:$P(A_1A_2\cdots A_n) = P(A_1)P(A_2|A_1)P(A_3|A_1A_2)\cdots P(A_n|A_1A_2\cdots A_{n-1})$
        \item 全概率公式:$P(A) = \sum_{i} P(A|B_i)P(B_i)$,其中 $\{B_i\}$ 为 $\Omega$ 的一个分割
        \item 贝叶斯公式:$P(B_k|A) = \frac{P(A|B_k)P(B_k)}{\sum_{i} P(A|B_i)P(B_i)}$,其中 $\{B_i\}$ 为 $\Omega$ 的一个分割
    \end{enumerate}
\end{proposition}

可通过如下的例子来理解条件概率

\begin{example}
    某病误诊率为 $5\%$,记 $A = \{\text{验血为阳性}\}$,$B = \{\text{患病}\}$,则 $P(\overline{A}|B) = 0.95$,$P(A|\overline{B}) = 5\%$,若患病率 $0.5\%$,即 $P(B)=0.005$,求 $P(B|A)$
\end{example}

\begin{solution}
    由贝叶斯公式
    \[ P(B|A) = \frac{P(A|B)P(B)}{P(A|B)P(B) + P(A|\overline{B})P(\overline{B})} = \frac{0.95\times 0.005}{0.95\times 0.005 + 0.05\times 0.995} \approx 0.087 \]
    也就是说,验血为阳性的人中,患病的概率为 $8.7\%$
\end{solution}

\subsection{独立性}

\begin{definition}[独立性]
    两个事件:$A$ 和 $B$ 独立可用下式表示
    \begin{equation}
        \text{独立} \Leftrightarrow P(AB) = P(A)P(B) \Leftrightarrow P(A|B) = P(A) \Leftrightarrow P(B|A) = P(B)
    \end{equation}
    有限多个事件:$n$ 个事件 $A_1,A_2,\ldots,A_n$ 独立可用下式表示\footnote{共有 $2^n-n-1$ 个式子}
    \begin{equation}
        \text{独立} \Leftrightarrow \forall i_1,i_2,\ldots,i_k\in\{1,2,\ldots,n\}, P(A_{i_1}A_{i_2}\cdots A_{i_k}) = P(A_{i_1})P(A_{i_2})\cdots P(A_{i_k})
    \end{equation}
    无限多个事件:设 $\{A_t\}\subseteq\mathcal{F}$,其中 $t\in T$,则定义 $\{A_t\}$ 独立为
    \begin{equation}
        \text{独立} \Leftrightarrow \forall n\in\mathbb{N},\forall t_1,t_2,\ldots,t_n\in T, A_{t_1}A_{t_2}\cdots A_{t_n} \text{ 独立}
    \end{equation}        
\end{definition}

独立事件有如下性质

\begin{proposition}
    设 $n$ 个事件 $A_1,A_2,\ldots,A_n$ 独立,则
    \begin{equation}
        P(A_1A_2\cdots A_n) = P(A_1)P(A_2)\cdots P(A_n)
    \end{equation}
    从而进一步有
    \begin{equation}
        P(\cup_{i=1}^n A_i) = 1 - \prod_{i=1}^n (1-P(A_i))
    \end{equation}
\end{proposition}

\appendix


\bibliographystyle{unsrt}
{\footnotesize\bibliography{./library}}


\end{document}
