\documentclass[12pt,a4paper]{amsart}
\usepackage[UTF8]{ctex}
\usepackage{preamble}

% 定义 example 环境
\newcounter{example}[section] % 定义 example 计数器
\newenvironment{example}{%
    \refstepcounter{example}% 增加计数器
    \par\medskip\noindent
    \textbf{Example \theexample:}\ %
}{%
    \par\medskip
}

% 定义 solution 环境
\newcounter{solution}[section] % 定义 solution 计数器
\newenvironment{solution}{%
    \refstepcounter{solution}% 增加计数器
    \par\medskip\noindent
    \textbf{Solution \thesolution:}\ %
}{%
    \par\medskip
}

\title{Chapter 2 随机变量}

\begin{document}

\maketitle\cite{杨振明2007}

\section{随机变量}

\begin{definition}[随机变量]
    设 $(\Omega, \mathcal{F}, P)$ 为概率空间,$\xi = \xi(\omega): \Omega\to\R$,若
    \begin{equation}
        \{\omega: \xi(\omega) < x\} \in \mathcal{F}, \quad \forall x \in \R
    \end{equation}
    则称 $\xi$ 为随机变量
\end{definition}

注:$\xi(\omega)$ 可简写为 $\xi$,$\xi(\omega) < x$ 可简写为 $\xi < x$

随机变量的逆变换有如下性质

\begin{proposition}
    设 $\xi$ 为随机变量,其逆变换为 $\xi^{-1}$,则
    \begin{enumerate}
        \item $\xi^{-1}(\R)=\Omega$
        \item $\forall B\subseteq C \Rightarrow \xi^{-1}(B) \subseteq \xi^{-1}(C)$
        \item $\xi^{-1}(\overline{B}) = \overline{\xi^{-1}(B)}$
        \item $\xi$ 的 Borel 函数仍为随机变量
    \end{enumerate}
\end{proposition}

\begin{proposition}[随机变量的结构]
    \begin{enumerate}
        \item $(\Omega, \mathcal{F}, P)$ 中 $E\in F$ 的示性函数 $\mathbb{1}_E(\omega)$ 为随机变量
        \item 设 $(\Omega, \mathcal{F}, P)$ 为 R.V. $\Leftrightarrow$ $\exists$ 简单随机变量列 $\{\xi_n\}_{n\geq 1}$ $s.t. \lim\limits_{n\to\infty}\xi_n(\omega)=\xi(\omega), \forall\omega\in\Omega$(此处极限为逐点收敛)
    \end{enumerate}
\end{proposition}

\section{随机分布(分布)}

\begin{definition}[分布]
    $(\Omega, \mathcal{F}, P)$ 为概率空间,$\mathbb{F}(B) = P\{\xi^{-1}(B)\} = P\{\xi\in B\}, \forall B\in\mathcal{B}$,则 $\mathbb{F}$ 为一个概率测度,称作 $\xi$ 的概率分布
\end{definition}

$\mathbb{F}$ 刻画了 $\xi$ 的分布规律

\begin{definition}[相空间]
    $(\R, \mathcal{B}, \mathbb{F})$ 为 $\xi$ 的相空间,$\mathbb{F}$ 与 $\xi$ 有关
\end{definition}

\begin{definition}[分布函数]
    $F(x) = \mathbb{F}((-\infty, x]) = P\{\xi\leq x\}, \forall x\in\R$,称 $F(x)$ 为 $\xi$ 的分布函数
\end{definition}

分布函数具有以下性质

\begin{proposition}
    \begin{enumerate}
        \item $F(x)$ 单调不减,即 $x_1\leq x_2 \Rightarrow F(x_1)\leq F(x_2)$   
        \item $F(x)$ 左连续,即 $x_0\in\R \Rightarrow \lim\limits_{x\to x_0^-}F(x)=F(x_0)$
        \item $F(-\infty) = \lim\limits_{x\to-\infty}F(x)=0, F(+\infty) = \lim\limits_{x\to+\infty}F(x)=1$
    \end{enumerate}
\end{proposition}

\begin{definition}[离散型]
    $(\Omega, \mathcal{F}, P)$ 为概率空间,$\xi$ 为随机变量,若 $\exists \{x_k\}, \{p_k\}$ 使得
    \begin{equation}
        p_k \geq 0, \sum_{k}p_k = 1, P\{\xi = x_k\} = p_k, \forall k
    \end{equation}
    则称 $\xi$ 为离散型随机变量,其密度阵
    \begin{equation}
        \begin{pmatrix}
            x_1 & x_2 & \cdots & x_n \\
            p_1 & p_2 & \cdots & p_n
        \end{pmatrix}
    \end{equation}
\end{definition}

\begin{definition}[连续型]
    $(\Omega, \mathcal{F}, P)$ 为概率空间,$\xi$ 为随机变量,若 $\exists p(x)$ 使得
    \begin{equation}
        p(x) \geq 0, \int_{-\infty}^{+\infty}p(x)dx = 1, P\{\xi=a\} = 0, \forall a
    \end{equation}
    则称 $\xi$ 为连续型随机变量,其密度函数为 $p(x)$
\end{definition}

不管是什么类型的随机变量,我们都有

\begin{proposition}[Lebesgue 分解]
    任意分布函数 $F(x)$ 可分解为
    \begin{equation}
        F(x) = c_1F_1(x) + c_2F_2(x) + c_3F_3(x)
    \end{equation}
    其中 $F_1(x)$ 为纯条约分布函数,$F_2(x)$ 为连续型分布函数,$F_3(x)$ 为奇异分布函数
\end{proposition}

\section{一些典型分布}

\subsection{二项分布}

$\xi\sim B(n,p)$

\begin{definition}[二项分布]
    设 $n$ 次独立重复试验,每次试验成功的概率为 $p$,失败的概率为 $1-p$,则 $\xi$ 为成功次数,$\xi\sim B(n,p)$
\end{definition}

\begin{proposition}[二项分布的分布函数]
    \begin{equation}
        P\{\xi=k\} = \binom{n}{k}p^k(1-p)^{n-k}
    \end{equation}
\end{proposition}

\subsection{几何分布}

$\xi\sim G(p)$

\begin{definition}[几何分布]
    设独立重复试验,每次试验成功的概率为 $p$,失败的概率为 $1-p$,则 $\xi$ 为第一次成功所需的试验次数,$\xi\sim G(p)$
\end{definition}

\begin{proposition}[几何分布的分布函数]
    \begin{equation}
        P\{\xi=k\} = (1-p)^{k-1}p
    \end{equation}
\end{proposition}

\subsection{Pascal 分布(负二项分布)}

$\xi\sim F(n,p)$

\begin{definition}[Pascal 分布]
    设独立重复试验,每次试验成功的概率为 $p$,失败的概率为 $1-p$,则 $\xi$ 为第 $n$ 次成功所需的试验次数,$\xi\sim F(n,p)$
\end{definition}

\begin{proposition}[Pascal 分布的分布函数]
    \begin{equation}
        P\{\xi=k\} = \binom{k-1}{n-1}p^n(1-p)^{k-n}
    \end{equation}
\end{proposition}

\subsection{Poisson 分布}

$\xi\sim P(\lambda)$

\begin{definition}[Poisson 分布]
    设独立重复试验,每次试验成功的概率为 $\lambda/n$,失败的概率为 $1-\lambda/n$,则 $\xi$ 为成功次数,$\xi\sim P(\lambda)$
\end{definition}

\begin{proposition}[Poisson 分布的分布函数]
    \begin{equation}
        P\{\xi=k\} = \frac{\lambda^k}{k!}e^{-\lambda}
    \end{equation}
\end{proposition}

\subsection{正态分布}

$\xi\sim N(\mu, \sigma^2)$

\begin{definition}[正态分布]
    设 $\xi$ 服从正态分布 $N(\mu, \sigma^2)$,则其密度函数为
    \begin{equation}
        p(\xi) = \phi_{\mu,\sigma} (\xi) = \frac{1}{\sqrt{2\pi}\sigma}e^{-\frac{(\xi-\mu)^2}{2\sigma^2}}
    \end{equation}
\end{definition}

\begin{proposition}[正态分布的分布函数]
    \begin{equation}
        F(x) = \frac{1}{\sqrt{2\pi}\sigma}\int_{-\infty}^{x}e^{-\frac{(t-\mu)^2}{2\sigma^2}}dt
    \end{equation}
\end{proposition}

\subsection{Gamma 分布}

$\xi\sim \Gamma(\lambda, r)$

\begin{definition}[Gamma 分布]
    设 $\xi$ 服从 Gamma 分布 $\Gamma(\lambda, r)$,其中 $\lambda>0, r>0$,则其密度函数为
    \begin{equation}
        p(\xi) = \frac{\lambda^r}{\Gamma(r)}\xi^{r-1}e^{-\lambda\xi}
    \end{equation}
\end{definition}

\begin{proposition}[Gamma 分布的分布函数]
    \begin{equation}
        F(x) = \frac{\lambda^r}{\Gamma(r)}\int_{0}^{x}t^{r-1}e^{-\lambda t}dt
    \end{equation}
\end{proposition}

\subsection{指数分布}

Gamma 分布中 $r=1$ 的情形,$\xi\sim E(\lambda)$

\begin{definition}[指数分布]
    设 $\xi$ 服从指数分布 $E(\lambda)$,则其密度函数为
    \begin{equation}
        p(\xi) = \lambda e^{-\lambda\xi}
    \end{equation}
\end{definition}

\begin{proposition}[指数分布的分布函数]
    \begin{equation}
        F(x) = \int_{0}^{x}\lambda e^{-\lambda t}dt
    \end{equation}
\end{proposition}

\subsection{各种分布的关系}

\begin{figure}[htbp]
    \centering
    \includegraphics[width=0.8\textwidth]{./src/1.png}
    \caption{各种分布的关系}
    \label{fig:1}
\end{figure}

\section{多维概率分布}

\subsection{联合分布}

\begin{definition}[联合分布函数]
    设 $(\Omega_1, \mathcal{F}_1, P_1)$ 与 $(\Omega_2, \mathcal{F}_2, P_2)$ 为概率空间,$\xi_1, \xi_2$ 为随机变量,$\xi_1: \Omega_1\to\R, \xi_2: \Omega_2\to\R$,则 $\xi_1, \xi_2$ 的联合分布函数为
    \begin{equation}
        F(x_1, x_2) = P\{\xi_1 < x_1, \xi_2 < x_2\}
    \end{equation}
\end{definition}

\begin{proposition}[联合分布函数的性质]
    \begin{enumerate}
        \item $F(x_1, x_2)$ 对 $x_1, x_2$ 单调非降
        \item $F(x_1, x_2)$ 对 $x_1, x_2$ 左连续
        \item $F(-\infty, x_2) = F(x_1, -\infty) = 0$
        \item $F(+\infty, +\infty) = 1$
        \item $\Delta F = F(b_1, b_2) - F(a_1, b_2) - F(b_1, a_2) + F(a_1, a_2) \geq 0$,且 $\Delta F$ 表示落入 $[a_1, a_2)\times[b_1, b_2)$ 的概率
    \end{enumerate}
\end{proposition}

\begin{definition}[离散型联合分布]
    $(\xi, \eta)$ 有至多可列对 $(x_i, y_j)$,其联合分布为
    \begin{equation}
        P\{\xi=x_i, \eta=y_j\} = p_{ij} \geq 0, \sum_{i,j}p_{ij} = 1
    \end{equation}
\end{definition}

\begin{definition}[连续型联合分布]
    \begin{enumerate}
        \item $(\xi, \eta)$ 有密度函数 $p(x, y)$,其分布函数为
        \begin{equation}
            F(x,y) = \int_{-\infty}^{x}\int_{-\infty}^{y}p(u, v)dudv
        \end{equation}
        其中 $p(x, y) \geq 0, \int_{-\infty}^{+\infty}\int_{-\infty}^{+\infty}p(x, y)dxdy = 1$ 为密度函数
        \item 对任意的 Borel 集合 $B\in\mathcal{B}^2$,有
        \begin{equation}
            P\{(\xi, \eta)\in B\} = \iint_Bp(x, y)dxdy
        \end{equation}
    \end{enumerate}
\end{definition}

\begin{definition}[边缘分布]
    设 $(\xi, \eta)$ 有联合分布 $F(x, y)$,则 $\xi$ 的边缘分布为
    \begin{equation}
        F_1(x) = P\{\xi < x\} = F(x, +\infty)
    \end{equation}
    $\eta$ 的边缘分布为
    \begin{equation}
        F_2(y) = P\{\eta < y\} = F(+\infty, y)
    \end{equation}
\end{definition}

\begin{definition}[离散型边缘分布]
    \begin{equation}
        P\{\xi=x_i\} = \sum_{j}p_{ij}, P\{\eta=y_j\} = \sum_{i}p_{ij}
    \end{equation}
\end{definition}

\begin{definition}[连续型边缘分布]
    \begin{equation}
        \begin{aligned}
            p_1(x) = \int_{-\infty}^{+\infty}p(x, y)dy \\
            p_2(y) = \int_{-\infty}^{+\infty}p(x, y)dx
        \end{aligned}
    \end{equation}
    \begin{equation}
        \begin{aligned}
            F_1(x) = \int_{-\infty}^{x}p_1(u)du = \int_{-\infty}^{x}\int_{-\infty}^{+\infty}p(u, v)dudv \\
            F_2(y) = \int_{-\infty}^{y}p_2(v)dv = \int_{-\infty}^{y}\int_{-\infty}^{+\infty}p(u, v)dvdu
        \end{aligned}
    \end{equation}
\end{definition}

\subsection{典型的二维分布}

\begin{definition}[二维均匀分布]
    设 $(\xi, \eta)$ 有联合密度函数
    \begin{equation}
        p(x, y) = \begin{cases}
            \frac{1}{S}, & (x, y)\in D \\
            0, & \text{其他}
        \end{cases}
    \end{equation}
    其中 $D\in\mathcal{B}^2$ 为平面上的区域,满足 $0 < m(D) < +\infty$,$S = m(D)$ 为 $D$ 的面积,则称 $(\xi, \eta)$ 服从二维均匀分布
\end{definition}

\begin{definition}[二维正态分布]
    设 $(\xi, \eta)$ 有联合密度函数
    \begin{equation}
        p(x, y) = \frac{1}{2\pi\sigma_1\sigma_2\sqrt{1-\rho^2}}\exp\left\{-\frac{1}{2(1-\rho^2)}\left[\frac{(x-\mu_1)^2}{\sigma_1^2}-2\rho\frac{(x-\mu_1)(y-\mu_2)}{\sigma_1\sigma_2}+\frac{(y-\mu_2)^2}{\sigma_2^2}\right]\right\}
    \end{equation}
    其中 $\mu_1, \mu_2, \sigma_1, \sigma_2, \rho$ 为常数,$\sigma_1, \sigma_2 > 0, \rho\in(-1, 1)$,则称 $(\xi, \eta)$ 服从二维正态分布,记作 $(\xi, \eta)\sim N(\mu_1, \mu_2, \sigma_1^2, \sigma_2^2, \rho)$ 或 $\begin{pmatrix} \xi \\ \eta \end{pmatrix} \sim N \left( \begin{pmatrix} \mu_1 \\ \mu_2 \end{pmatrix}, \begin{pmatrix} \sigma_1^2 & \rho\sigma_1\sigma_2 \\ \rho\sigma_1\sigma_2 & \sigma_2^2 \end{pmatrix} \right)$
\end{definition}

\begin{proposition}
    二维正态分布的性质
    \begin{enumerate}
        \item $(\xi, \eta)$ 的边缘分布为
        \begin{equation}
            \begin{aligned}
                \xi &\sim N(\mu_1, \sigma_1^2) \\
                \eta &\sim N(\mu_2, \sigma_2^2)
            \end{aligned}
        \end{equation}
        \item $(\xi, \eta)$ 的条件分布为
        \begin{equation}
            \begin{aligned}
                \xi|\eta = y &\sim N\left(\mu_1+\rho\frac{\sigma_1}{\sigma_2}(y-\mu_2), \sigma_1^2(1-\rho^2)\right) \\
                \eta|\xi = x &\sim N\left(\mu_2+\rho\frac{\sigma_2}{\sigma_1}(x-\mu_1), \sigma_2^2(1-\rho^2)\right)
            \end{aligned}
        \end{equation}
        \item $(\xi, \eta)$ 的独立性与 $\rho$ 无关
    \end{enumerate}
\end{proposition}

\section{随机变量的独立性}\footnote{关于独立性,第一章涉及事件的独立性,本章则主要探讨随机变量的独立性,注意区分}

\subsection{条件分布}

\begin{definition}[条件分布]
    设 $(\xi, \eta)$ 有联合分布 $F(x, y)$,设 $B\in \mathcal{B}$,则 $\xi$ 在条件 $\eta\in B$ 下的条件分布为
    \begin{equation}
        F(x|B) = P\{\xi < x|\eta\in B\} = \frac{P\{\xi < x, \eta\in B\}}{P\{\eta\in B\}}
    \end{equation}
\end{definition}

\begin{definition}[离散型条件分布]
    \begin{equation}
        P\{\xi=x_i|\eta=y_j\} = \frac{P\{\xi=x_i, \eta=y_j\}}{P\{\eta=y_j\}}
    \end{equation}
\end{definition}

\begin{definition}[连续型条件分布]
    \begin{equation}
        \begin{aligned}
            p(x|y) &= \frac{p(x, y)}{p_2(y)} \\
            F(x|y) &= \int_{-\infty}^{x}p(u|y)du
        \end{aligned}
    \end{equation}
\end{definition}

\subsection{随机变量独立性}

根据条件分布的概念,我们可以定义随机变量的独立性

\begin{definition}[两个随机变量的独立性]
    设 $(\xi, \eta)$ 有联合分布 $F(x, y)$,$\xi$ 在条件 $\eta=y$ 下的条件分布为 $F(x|y)$,若
    \begin{equation}
        F(x, y) = F_1(x)F_2(y)
    \end{equation}
    则称 $\xi$ 与 $\eta$ 独立
\end{definition}

\begin{definition}[有限个随机变量的独立性]
    设 $(\xi_1, \xi_2, \cdots, \xi_n)$ 有联合分布 $F(x_1, x_2, \cdots, x_n)$,若
    \begin{equation}
        F(x_1, x_2, \cdots, x_n) = F_1(x_1)F_2(x_2)\cdots F_n(x_n)
    \end{equation}
    则称 $\xi_1, \xi_2, \cdots, \xi_n$ 独立
\end{definition}

\begin{definition}[无限个随机变量的独立性]
    设 $(\xi_1, \xi_2, \cdots)$ 有联合分布 $F(x_1, x_2, \cdots)$,若其中任意有限个随机变量独立,则称 $\xi_1, \xi_2, \cdots$ 独立
\end{definition}

\begin{proposition}\footnote{由此性质知,多个变量独立可以推出其中两两独立,但反之不成立}
    关于有限个变量独立,以下两命题等价
    \begin{enumerate}
        \item $\xi_1, \xi_2, \cdots, \xi_n$ 独立
        \item $\forall i_1, i_2, \cdots, i_k, k\leq n$,$\xi_{i_1}, \xi_{i_2}, \cdots, \xi_{i_k}$ 独立
    \end{enumerate}
\end{proposition}

\begin{proposition}
    关于两个离散变量独立,以下两命题等价
    \begin{enumerate}
        \item $\xi, \eta$ 独立
        \item $\forall x_i, y_j, P\{\xi=x_i, \eta=y_j\} = P_1\{\xi=x_i\}P_2\{\eta=y_j\}$(联合分布为边缘分布之积)
    \end{enumerate}
\end{proposition}

\begin{proposition}
    关于 $\R$ 上的 Borel 函数 $f_1, f_2$,第一个命题成立可推出第二个命题成立
    \begin{enumerate}
        \item $\xi, \eta$ 独立
        \item $\forall \text{Borel 函数} f_1, f_2, f_1(\xi), f_2(\eta)$ 独立
    \end{enumerate}
\end{proposition}

\section{随机变量函数的分布}

如下计算一个随机变量在连续单调函数下的分布

\begin{proposition}
    设 $\xi$ 为随机变量,密度函数 $p_{\xi}(x)$,$\eta = f(\xi)$,若 $y = f(x)$ 在 $(a, b)$ 上严格单调、连续,则反函数 $x = f^{-1}(y)$ 存在且连续,此时 $\eta$ 也为随机变量,其密度函数为
    \begin{equation}
        p_{\eta}(y) = p_{\xi}(f^{-1}(y))\left|\frac{d}{dy}f^{-1}(y)\right|
    \end{equation}
\end{proposition}

\begin{example}
    设 $\xi\sim N(0, 1)$,求 $\eta = \xi^2$ 的分布
\end{example}

\begin{solution}
    若 $y < 0$,则不存在 $\xi$ 使得 $\xi^2 = y$,故 $p_{\eta}(y) = 0$ \\
    下面考虑 $y \geq 0$ 的情况 \\
    $f(x) = x^2$ 在 $[0, +\infty)$ 上严格单调、连续,反函数 $f^{-1}(y) = +\sqrt{y}$ 存在且连续 \\
    从而
    \begin{equation}
        p_{\eta, 1}(y) = p_{\xi}(+\sqrt{y})\left|\frac{d}{dy}+\sqrt{y}\right| = \frac{1}{\sqrt{2\pi}}e^{-\frac{y}{2}}\frac{1}{2\sqrt{y}}
    \end{equation}
    由对称性,对于 $x < 0$
    \begin{equation}
        p_{\eta, 2}(y) = p_{\xi}(-\sqrt{y})\left|\frac{d}{dy}-\sqrt{y}\right| = \frac{1}{\sqrt{2\pi}}e^{-\frac{y}{2}}\frac{1}{2\sqrt{y}}
    \end{equation}
    故 $\eta$ 的分布为
    \begin{equation}
        p_{\eta}(y) = \begin{cases}
            \frac{1}{\sqrt{2\pi y}}e^{-\frac{y}{2}}, & y > 0 \\
            0, & y \leq 0
        \end{cases}
    \end{equation}
\end{solution}

对于连续型随机变量在函数变换下的分布,我们有若干性质

\begin{proposition}
    设 $\xi, \eta$ 为两个随机变量,联合密度函数为 $p_{\xi, \eta}(x, y)$($\xi$ 和 $\eta$ 不一定独立)
    \begin{enumerate}
        \item 和的密度
        \begin{equation}
            \begin{aligned}
                p_{\xi+\eta}(z) &= \int_{-\infty}^{+\infty}p_{\xi, \eta}(x, z-x)dx = \int_{-\infty}^{+\infty}p_{\xi, \eta}(z-y, y)dy \\
                F_{\xi+\eta}(z) &= \int_{-\infty}^{+\infty}F_{\eta}(z-x)dF_{\xi}(x) = \int_{-\infty}^{+\infty}F_{\xi}(z-y)dF_{\eta}(y)
            \end{aligned}
        \end{equation}
        \item 差的密度
        \begin{equation}
            \begin{aligned}
                p_{\xi-\eta}(z) &= \int_{-\infty}^{+\infty}p_{\xi, \eta}(x, x-z)dx = \int_{-\infty}^{+\infty}p_{\xi, \eta}(z+y, y)dy \\
                F_{\xi-\eta}(z) &= \int_{-\infty}^{+\infty}F_{\eta}(x-z)dF_{\xi}(x) = \int_{-\infty}^{+\infty}F_{\xi}(z+y)dF_{\eta}(y)
            \end{aligned}
        \end{equation}
        \item 积的密度
        \begin{equation}
            \begin{aligned}
                p_{\xi\eta}(z) &= \int_{-\infty}^{+\infty}p_{\xi, \eta}(x, \frac{z}{x})\frac{1}{|x|}dx = \int_{-\infty}^{+\infty}p_{\xi, \eta}(z\cdot y, y)\frac{1}{|y|}dy \\
                F_{\xi\eta}(z) &= \int_{-\infty}^{+\infty}F_{\eta}(\frac{z}{x})dF_{\xi}(x) = \int_{-\infty}^{+\infty}F_{\xi}(z\cdot y)dF_{\eta}(y)
            \end{aligned}
        \end{equation}
        \item 商的密度
        \begin{equation}
            \begin{aligned}
                p_{\frac{\xi}{\eta}}(z) &= \int_{-\infty}^{+\infty}p_{\xi, \eta}(x, \frac{x}{z})\frac{1}{|x|}dx = \int_{-\infty}^{+\infty}p_{\xi, \eta}(z\cdot y, y)\frac{1}{|y|}dy \\
                F_{\frac{\xi}{\eta}}(z) &= \int_{-\infty}^{+\infty}F_{\eta}(\frac{x}{z})dF_{\xi}(x) = \int_{-\infty}^{+\infty}F_{\xi}(z\cdot y)dF_{\eta}(y)
            \end{aligned}
        \end{equation}
    \end{enumerate}
\end{proposition}

\appendix


\bibliographystyle{unsrt}
{\footnotesize\bibliography{./library}}


\end{document}
